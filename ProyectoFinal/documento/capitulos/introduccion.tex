\section{Motivación}
El ransomware es una forma específica de malware cuyo objetivo principal es bloquear el acceso a los datos de una organización o individuo, generalmente mediante técnicas de cifrado. Una vez que los archivos han sido cifrados, el atacante demanda el pago de un rescate (habitualmente en criptomonedas) a cambio de proporcionar la clave necesaria para restaurar la información secuestrada\cite{NISTransomware2023}.

Esta clase de ataques se ha convertido en una amenaza persistente y cada vez más frecuente en el panorama actual de la ciberseguridad. Según el informe anual de la empresa especializada en ciberseguridad Cyberint\cite{Bleih2025}, el pasado año 2024 se reportaron, a nivel global, 5.414 ataques de ransomware alrededor del mundo lo que supone un incremento del 11\% respecto al año 2023. Es también destacable el hecho de que un tercio del total de los ataques se realizaron en el último trimestre del año, lo que puede indicar una tendencia de aumento de ataques de esta clase en el presente año 2025.

Dado que este tipo de ataques no solo persisten, sino que van en aumento, resulta imprescindible desarrollar mecanismos capaces de detectarlos y prevenirlos sin que ello afecte negativamente al rendimiento del sistema ni se convierta en un cuello de botella. Para poder desplegar un sistema de detección y respuesta ante ransomware sin afectar al renidimiento del sistema, es fundamental analizar dónde implementar dicho sistema, tanto a nivel lógico como físico.

Un aspecto fundamental a considerar es en qué espacio del sistema operativo debe desplegarse la medida de seguridad. En el artículo de Parola et al. (2023) \cite{Parola2023}, se compara el rendimiento de la gestión de paquetes en tres escenarios: desde el espacio de usuario mediante el flujo tradicional, desde el espacio de usuario utilizando la herramienta DPDK —que opera mediante técnicas de polling para evitar la intervención del kernel— y, finalmente, desde el propio espacio del kernel.

Aunque DPDK presenta la menor latencia, el estudio concluye que, en términos de equilibrio entre rendimiento y consumo de recursos, el enfoque más eficiente es el que se implementa directamente en el espacio del kernel. Este equilibrio sugiere que podría resultar especialmente interesante explorar el desarrollo de mecanismos de detección de ransomware a nivel de kernel, ya que permitiría una supervisión más cercana al sistema sin comprometer significativamente el rendimiento.

Otro aspecto relevante a considerar es el nodo físico de la red en el que debe implementarse el sistema de detección. La práctica más común consiste en desplegar el software de detección directamente en el servidor que se desea proteger —por ejemplo, el servidor A—. Sin embargo, teniendo en cuenta el crecimiento sostenido de las redes, con una tasa de crecimiento compuesta del 24\% según un informe de Cisco (2021) \cite{Cisco2021}, este enfoque puede resultar cada vez menos eficiente.

El aumento del tráfico puede derivar en una mayor exposición a ataques, lo que incrementa la carga computacional que el servidor A debe asumir para protegerse a sí mismo. Por ello, puede resultar conveniente delegar parte de esta responsabilidad a otros componentes de la infraestructura de red, como las Tarjetas de Interfaz de Red (NIC, por sus siglas en inglés). Dentro de esta categoría, destacan especialmente las SmartNICs, que incorporan unidades de procesamiento dedicadas (DPUs) capaces de ejecutar tareas de seguridad de forma autónoma, descargando así al servidor principal.

\section{Objetivos}

\section{Estructura del documento}
