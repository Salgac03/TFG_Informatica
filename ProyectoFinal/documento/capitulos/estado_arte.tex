\section{Conjunto de Datos}
En este apartado se describe el conjunto de datos empleado para el entrenamiento del modelo de árbol de decisión, diseñado con el objetivo de detectar posibles comunicaciones asociadas a ransomware.

El conjunto de datos utilizado proviene del trabajo realizado por Eduardo Berrueta et al. entre los años 2015 y 2022\cite{qnyn-q136-20}. Este dataset contiene capturas de tráfico en formato .pcap correspondientes a setenta familias distintas de ransomware, con un volumen total superior a 60 GB con los datos comprimidos.

Debido a las limitaciones de almacenamiento y capacidad de cómputo disponibles, así como al alcance definido para este trabajo, he optado por trabajar únicamente con dos de las muestras incluidas en el conjunto de datos original, que serán los que trataremos en profundidad en este apartado.

Pero antes de tratar estos dos subconjuntos de datos en concreto, cabe hacer una mención al artículo de Eduardo Berrueta y su equipo\cite{9050526}. En él explican que la motivación para crear el dataset proviene de la necesidad de estandarizar recursos a la hora de evaluar la fiabilidad de las herramientas de detección de ransomware, ya que habitualmente cada fabricante o investigador utiliza su propio conjunto de datos. Al ser estos conjuntos muy específicos y reducidos, las comparativas resultantes tienden a ser poco realistas. Con su trabajo, Berrueta pretende aportar un dataset completo para pruebas de detección de ransomware que contribuya a resolver estas problemáticas.


\section{Artículos Similares y Relacionados}
