\section{Conclusiones}

En esta sección se presentan las conclusiones del presente trabajo, con el objetivo de valorar si, tal y como se estableció en los objetivos planteados, el uso de clasificadores sencillos, como los árboles de decisión, aplicados como filtros preliminares en tareas de ciberseguridad, puede introducir un impacto apreciable sobre el rendimiento de la red. Para ello, se tienen en cuenta los distintos escenarios de experimentación definidos, así como las métricas de rendimiento seleccionadas, analizando la viabilidad de este enfoque desde un punto de vista tanto técnico como práctico.

\section{Futuros Trabajos}

A partir del presente trabajo se pueden plantear diversas líneas de investigación. En primer lugar, sería interesante reproducir estos experimentos en una Smart NIC real, con el objetivo de evaluar si los árboles de decisión implementados mediante XDP constituyen una opción viable para ser utilizados como filtro preliminar en entornos de red de alto rendimiento.

Una segunda línea de investigación consistiría en realizar pruebas similares empleando árboles de decisión basados en la tecnología DPDK en lugar de XDP, con el fin de comparar la eficiencia y la idoneidad de ambas tecnologías para el filtrado de tráfico en tiempo real.

Finalmente, y como una línea más avanzada, se podría explorar el impacto de la utilización de múltiples árboles de decisión, conformando un Random Forest, sobre el rendimiento del sistema. Esta aproximación permitiría evaluar si compensa la mayor complejidad computacional frente a las ventajas de disponer de árboles especializados para distintas características del tráfico, posibilitando así la creación de un filtro más especializado contra ransomware o, incluso, un sistema más generalista capaz de detectar múltiples tipos de ciberamenazas mediante la combinación de árboles de decisión dedicados a cada una de ellas.
