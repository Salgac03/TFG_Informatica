\section{Conclusiones}

En esta sección se presentan las conclusiones del presente trabajo, con el objetivo de valorar si, tal y como se estableció en los objetivos planteados, el uso de clasificadores sencillos, como los árboles de decisión, aplicados como filtros preliminares en tareas de ciberseguridad, puede introducir un impacto apreciable sobre el rendimiento de la red. Para ello, se consideran los distintos escenarios de experimentación definidos y las métricas seleccionadas, evaluando la viabilidad de este enfoque desde un punto de vista tanto técnico como práctico.

En primer lugar, se valida que el entorno experimental utilizado es representativo y suficientemente robusto para analizar mecanismos de filtrado en línea, descartando que los efectos observados se deban a deficiencias de la infraestructura de pruebas. Esto puede comprobarse a partir de la coherencia entre la tasa objetivo y la tasa medida de paquetes por segundo, tal y como se muestra en la figura \ref{measuredvstarget}. Asimismo, se confirma que la distinción entre pérdida aparente y pérdida real resulta esencial para interpretar correctamente el comportamiento del filtro XDP, evitando atribuir al sistema pérdidas que forman parte de su funcionamiento normal.

En este sentido, los resultados confirman que el incremento observado en la pérdida global de paquetes cuando el filtro está activo no se debe a una degradación del rendimiento del sistema, sino al descarte intencionado de tráfico clasificado como malicioso. Esta conclusión se refuerza al analizar la pérdida real, representada en la figura \ref{lossreal}, donde se observa que el impacto sobre las pérdidas no intencionadas se mantiene contenido. De forma coherente con lo anterior, el mantenimiento de un rendimiento útil comparable en ambos escenarios (figura \ref{goodput}) permite concluir que la lógica de clasificación introducida no constituye un cuello de botella significativo en la ruta de datos. Adicionalmente, el análisis de consumo de recursos muestra que el coste computacional asociado al filtrado es reducido, manteniéndose estable tanto en uso de CPU como de memoria (figuras \ref{cpu} y \ref{ram}), lo que refuerza su idoneidad para despliegues prolongados.

En conjunto, los resultados experimentales permiten concluir que la implementación de un filtro basado en árboles de decisión mediante XDP es viable desde el punto de vista del rendimiento. El sistema mantiene niveles comparables de rendimiento útil, pérdida real y consumo de recursos tanto con el filtro activo como desactivado, incluso a tasas elevadas de paquetes por segundo. Por tanto, estos resultados respaldan el uso de clasificadores sencillos como filtros preliminares de seguridad en la ruta de datos, especialmente en entornos donde la latencia y el rendimiento son críticos, como SmartNICs o arquitecturas \textit{inline}.

\section{Futuros Trabajos}

A partir del presente trabajo se pueden plantear diversas líneas de investigación. En primer lugar, sería interesante reproducir estos experimentos en una Smart NIC real, con el objetivo de evaluar si los árboles de decisión implementados mediante XDP constituyen una opción viable para ser utilizados como filtro preliminar en entornos de red de alto rendimiento.

Una segunda línea de investigación consistiría en realizar pruebas similares empleando árboles de decisión basados en la tecnología DPDK en lugar de XDP, con el fin de comparar la eficiencia y la idoneidad de ambas tecnologías para el filtrado de tráfico en tiempo real.

Finalmente, y como una línea más avanzada, se podría explorar el impacto de la utilización de múltiples árboles de decisión, conformando un \textit{random forest}, sobre el rendimiento del sistema. Esta aproximación permitiría evaluar si compensa la mayor complejidad computacional frente a las ventajas de disponer de árboles especializados para distintas características del tráfico, posibilitando así la creación de un filtro más especializado contra ransomware o, incluso, un sistema más generalista capaz de detectar múltiples tipos de ciberamenazas mediante la combinación de árboles de decisión dedicados a cada una de ellas.
